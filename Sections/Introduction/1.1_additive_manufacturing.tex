\subsection{Additive manufacturing}
\label{sec:AM}
Additive manufacturing (AM) is a manufacturing technique were 3D parts are built by the addition of thin layers according to a computer-aided design (CAD) model. Due to problems with part accuracy, a limited variety of materials and mechanical performance of the part, the application of additive manufacturing has mainly been as a rapid prototyping (RP). In rapid prototyping, one aims to construct a engineering prototype of a part with as low lead time as possible. This enables the engineers and designers to perform tests and get a physical feeling of the part early on in the design process. As rapid prototyping usually value fast part delivery and complexity of the part over specific materials, accuracy or mechanical properties, rapid prototyping has proved to be an ideal area of application for additive manufacturing. However, as the additive manufacturing technology has evolved to create parts with higher accuracy and improved mechanical properties, its fields of applications are rapidly growing, \citep{Manufacturing}.

\nomenclature{AM}{Additive manufacturing}
\nomenclature{RP}{Rapid prototyping}
\nomenclature{3D}{Three dimensional}

\todo[inline, color=lightgray!40]
{
    Give an overview of additive manufacturing and contextualize it within the overall taxonomy of the field. Explore different kindof additive manufacturing and FEM in additive manufacturing.
    printing parameters: depositon speed, road size, deposition pattern deposition temperature, environment temperature

    \vspace{0.2cm}
}